The growth of Machine Learning has allowed us to solve new
problems within the field of Affective computing,
therefore it is important to use these new tools within this
domain to better solve problems.
Now many wearable technologies use sensors such as ECG in their devices.
Theses devices have multiple cores and using them efficiently for
affective computing applications is incredibly important.
We used the AMIGOs data set, with the EEG, ECG and GSR signals
we applied filtering, feature extraction, and dimentationality reduction
using 4 CPU cores.
Using this method we received a 3.6x performance improvement in running our
pre-processing stage.
From these features we classified them based on emotions
neutral, disgust, happiness, and etc.
We compared 3 different classifies Logistic Regression,
SVM, and KNN.
For certain emotions we were able to obtain an F1 score of 90\%.
Some emotions such as anger we had a harder time detecting
using theses methods and only obtained a score of 48\%.
We also combined these 3 models and used a voting classifier which provided
more reliable results.
We have shown that multi processing will benefit affective computing workloads.
Our results have also shown that combining multiple different signal types does
result in good classification results.

