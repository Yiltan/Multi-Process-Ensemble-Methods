\clearpage
\section{Background}
\label{sec:background}
\subsection{Sensors}
Affective computing often relies on gathering data from human sources.
Many different types of sensors exists for
gathing information which can be important in this area.
In this paper we will focus in the following three types of sensors;
Electromyography,
Electrocardiography,
and Electroencephalography.

\subsubsection{Electromyography (EMG)}
EMG sensors are used to measure muscle activity.
As muscles contract and relax,
electirical signals are sent via neurons.
These electrical signals then
can be recorded using various types of sensors.
One of the main types of sensors is sEMGs,
they are usually placed on the belly of the muscle
and are cleaned with alchol before palcing
on the surface of the skin.
Maximum voulentary contraction is used to normalize the signals
before begining recording.

\subsubsection{Electrocardiography (ECG)}
These types of sensors are very similar to EMGs
but they are used to measure the PQRST complex of a heart beat.
Each part of the PQRST complex
corresponds to a part of the heart's cycle as it pumps blood around the body.

\subsubsection{Electroencephalography (EEG)}
EEG is a method to record brain activity.
They measure the electrical signals which the brain generates.
As neurons in the brains send information,
electrical currents are generated and are then measured.
This procedure is difficult to setup,
gel is placed onto the scalp and then sensors are placed upon them.
Many different frequency bands can be filtered from these
signals to obtain different information about brain activity.

\subsubsection{Galvanic Skin Response (GSR)}
GSR are placed directly onto the skin of the particiapnts.
The state of a sweat gland varied the resitance of human skin.
Studies have shown that sweating is controlled by the sympathetic nervous system
therefore we can collect this information
to build models on the system's behaviour.

\subsection{Amigos Dataset}
The AMIGOS dataset was prepaired by research groups at
Queen Mary University of London, United Kingdom and
University of Trento, Italy \cite{AMIGOS:2018}.
This dataset allows for research to be conducted on
personality traits, mood and affect.
The dataset contains 40 participants who each watched 20 videos.
These 20 videos are split into two catagories; long and short.
Each of these catagories are furthsplit into
wheather the participant watched the video alone or within a group.
Data was collected using ECG, EEG, and GSR sensors.
The participants were also recorded but we will not use the video recordings
for our work.
These videos were labeled with:
valence, arousal, control, familiarity, like/dislike,
and selection of basic emotions.

\subsection{Machine Learning}
Feature Extraction - Currently done

PCA - Dimension reduction - Amir is using this

Models we will try and use:
SVM
KNN
Naive Bayes - not recognition to predict between two classes.

kkkkkkk




\subsection{Motivation}

