\section{Introduction}
Rescent advancements in
artificial inteligence (AI),
machine learning (ML),
and deep learning (DL)
we have seen a growth in
research using these tools to
solve problems in many different domains.
These domains include medical imaging \cite{Greenspan:2016},
wearable devices,
diverless cars,
and even scientific inverse problems \cite{laanait2019exascale}.
These tools have seen much growth in affective computing.

Affective computing is a system in which can detect the
emotional state of a human and use emotional inteligence
to interprate and interact with humans.
Understanding and interpriting human emotion is a
difficult task for computers;
its even difficult for some humans.
Using various sensing technologies such as electrocardiogram (ECG)
alows for detection of various human emotion.
Creating a mapping from these electrical signals to
distinct emotional catagories poses some problems as
these catagories such as happy or sad and not discreate.
There is often a spectrum between any two classes.

In this work we aim to understand the following questions.
\begin{itemize}
    \item Can we use EEG signals to correctly detect different emotions?
    \item Can we use ECG signals to correctly map them to valiance scores?
\end{itemize}

This paper is structured as follows;
Section \ref{sec:background} discusses the various types of sensors
which are used in affective computing and the dataset which we will be using.
%
Section \ref{sec:related_work} discusses the related work to this project
and their limitations.
%
In section \ref{sec:proposed_method} we outline our propsed method
and we then show our respective results in section \ref{sec:experimental_results}.
%
We present our final conclusion in section \ref{sec:conclusion}.
%
The division of work in this project is stated in section
\ref{sec:division_of_work}.
