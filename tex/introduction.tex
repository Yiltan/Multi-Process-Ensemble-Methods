\section{Introduction}
With recent advancements in
artificial intelligence (AI),
machine learning (ML),
and deep learning (DL)
we have seen a growth in
research using these tools to
solve problems in many different domains.
These domains include medical imaging \cite{Greenspan:2016},
wearable devices,
diver-less cars,
and even scientific inverse problems \cite{laanait2019exascale}.
These tools have seen much growth in affective computing.

Affective computing is a system in which it can detect the
emotional state of a human and use emotional intelligence
to interpret and interact with humans.
Understanding and interpreting human emotion is a
difficult task for computers;
its even difficult for some humans.
Using various sensing technologies such as electrocardiogram (ECG)
allows for detection of various human emotion.
Creating a mapping from these electrical signals to
distinct emotional categories poses some problems as
these categories such as happy or sad and not discrete.
There is often a spectrum between any two classes.

Recently Machine Learning has also been applied to wearable devices,
portable devices and IoT.
Many of these new products have multi-core chips.
Although they are less powerful than traditional desktop PC
CPUs there is still potential to utilise the hardware of these devices.

In this work we aim to understand the following questions.
\begin{itemize}
    \item Can we use EEG signals to correctly detect different emotions?
    \item Can we use multiprocessing to improve performance?
\end{itemize}

This paper is structured as follows;
Section \ref{sec:background} discusses the various types of sensors
which are used in affective computing and the data set which we will be using.
%
Section \ref{sec:related_work} discusses the related work to this project
and their limitations.
%
In section \ref{sec:proposed_method} we outline our proposed method
and we then show our respective results in section \ref{sec:experimental_results}.
%
We present our final conclusion in section \ref{sec:conclusion}.
%
The division of work in this project is stated in section
\ref{sec:division_of_work}.
