The growth of Machine Learning has allowed us to solve new
problems within the field of Affective computing,
therefore it is important to use the new Machine Learning tools within this
domain to solve problems more efficiently.
Now many wearable technologies use sensors such as ECG in their devices.
Theses devices have multiple cores and using them efficiently for
affective computing applications is incredibly important.
We used the AMIGOs data set, with the EEG, ECG and GSR signals, 
and applied filtering, feature extraction, and dimentationality reduction
in a parallel manner.
Using this method we received a 3.6x performance improvement in running our
pre-processing stage.
From these features we classified them based on the emotions:
neutral, disgust, happiness, surprise, anger, fear, and sadness.
We compared 4 different classifiers: Logistic Regression,
SVM, KNN, and AdaBoost.
For certain emotions we were able to obtain an F1 score of 91\%.
Some emotions such as anger we had a harder time detecting
using these methods and only obtained a score of 71\%.
We also combined these 4 models and used a voting classifier which provided
more reliable results.
We have shown that multi processing will benefit affective computing workloads.
Our results have also shown that combining multiple different signal types does
result in good classification results.

